\begin{tikzpicture}[scale=0.9]
  \begin{axis}[
    axis equal image = true,
    axis lines = middle,
    x label style={at={(axis description cs:0.5, -0.1)},
      anchor=north, text width = 5cm},
    y label style={at={(axis description cs:-0.15, 1)},
      anchor=north east, text width = 2.5 cm},
    xlabel={Porcentaje de la población ordenada por su renta ($p$)},
    ylabel={Porcentaje de la renta total},
    xmin=0, xmax=100, ymin=0, ymax=100,
    ticklabel style = {font=\footnotesize},
    xtick={0, 20, 40, 60, 80, 100},
    ytick={0, 20, 40, 60, 80, 100},
    % legend pos=south east,
    grid=major,
    ]
    % Línea de igualdad perfecta
    \addplot[domain=0:100, semithick, black!60!] {x} ;


    \addplot[very thick, blue!40!black, smooth]
    table[col sep=comma, x=p, y=L] {ineq-data.csv}
    node[pos=0.62, below right, yshift = -4pt, text=black, fill=white] {$L(p)$};
  \end{axis}
\end{tikzpicture}

%%% Local Variables:
%%% mode: LaTeX
%%% TeX-master: t
%%% End:
